\documentclass[UTF8]{ctexart}
\usepackage{geometry, CJKutf8}
\geometry{margin=1.5cm, vmargin={0pt,1cm}}
\setlength{\topmargin}{-1cm}
\setlength{\paperheight}{29.7cm}
\setlength{\textheight}{25.3cm}

% useful packages.
\usepackage{amsfonts}
\usepackage{amsmath}
\usepackage{amssymb}
\usepackage{amsthm}
\usepackage{enumerate}
\usepackage{graphicx}
\usepackage{multicol}
\usepackage{fancyhdr}
\usepackage{layout}
\usepackage{listings}
\usepackage{float, caption}

\lstset{
    basicstyle=\ttfamily, basewidth=0.5em
}

% some common command
\newcommand{\dif}{\mathrm{d}}
\newcommand{\avg}[1]{\left\langle #1 \right\rangle}
\newcommand{\difFrac}[2]{\frac{\dif #1}{\dif #2}}
\newcommand{\pdfFrac}[2]{\frac{\partial #1}{\partial #2}}
\newcommand{\OFL}{\mathrm{OFL}}
\newcommand{\UFL}{\mathrm{UFL}}
\newcommand{\fl}{\mathrm{fl}}
\newcommand{\op}{\odot}
\newcommand{\Eabs}{E_{\mathrm{abs}}}
\newcommand{\Erel}{E_{\mathrm{rel}}}

\begin{document}

\pagestyle{fancy}
\fancyhead{}
\lhead{何思壮, 3230103269}
\chead{数据结构与算法第四次作业}
\rhead{Oct.16th, 2024}

\section{测试程序的设计思路}

为方便测试,我增加了新的函数printList(),作用是打印出链表中除哨兵节点外的其他节点的data。其中,printList()包含了包含retrieve(),begin()和end()的测试。

我首先用初始化列表构造函数创建list1,再用拷贝构造函数将list1拷贝到list2,而后测试了size(),empty(),和clear(),同时测试了拷贝构造函数为深度赋值。

我用赋值运算符创建了list3,用list3测试了pop\_back(),pop\_front(),push\_back(),push\_front()功能。此外,用右值引用创建了list4,然后测试了移动调用函数,功能正常。

接上来是对运算符的测试,我首先对静态迭代器进行了一系列运算符的测试,然后用动态迭代器对列表进行写入操作,功能正常。

最后我进行了对insert(),erase()的测试,其中包括insert()的右值引用,一切正常。



\section{测试的结果}

测试结果一切正常。

我用 valgrind 进行测试,发现代码上没有发生内存泄露,但显示bash解释器会导致内存泄漏,我无法解决。

\section{(可选)bug报告}

我发现了一个 bug,触发条件如下:

\begin{enumerate}
    \item 首先令p为list.end(),再进行++p操作
    \item 然后尝试打印*p
    \item 报错
\end{enumerate}

据我分析,它出现的原因是:在各项功能中,end()作为哨兵节点,我们不会遍历或者调用到它,上述操作显然超出了list的范围导致了未定义行为。解决的方法是可以在重载运算符时增加条件,保证中间节点的操作不变,而哨兵节点无法被访问。复现较简单,已写。

\end{document}

%%% Local Variables: 
%%% mode: latex
%%% TeX-master: t
%%% End: 
